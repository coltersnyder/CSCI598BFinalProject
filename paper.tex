\documentclass[conference]{IEEEtran}
\IEEEoverridecommandlockouts
% The preceding line is only needed to identify funding in the first footnote. If that is unneeded, please comment it out.
\usepackage{biblatex}
\usepackage{amsmath,amssymb,amsfonts}
\usepackage{algorithmic}
\usepackage{graphicx}
\usepackage{textcomp}
\usepackage{xcolor}
\def\BibTeX{{\rm B\kern-.05em{\sc i\kern-.025em b}\kern-.08em
    T\kern-.1667em\lower.7ex\hbox{E}\kern-.125emX}}

\addbibresource{citations.bib}

\begin{document}

\title{Graphical Databases for IoT Device Management and Anomoly Detection}

\author{\IEEEauthorblockN{Colter Snyder}
\IEEEauthorblockA{\textit{Department of Computer Science} \\
\textit{Colorado School of Mines}\\
Golden, Colorado, USA \\
csnyder1@mines.edu}
}

\maketitle

\begin{abstract}
IoT devices have exploded in use over the past few years.
This has caused concerns over data privacy and security.
Many papers have been writen to detect IoT devices and catalog them,
however none use a graphical database in order to do so.
Using a graphical database provides many benefits over current systems
and allows for new questions to be answered.
\end{abstract}

\begin{IEEEkeywords}
IoT, graphical databases, databases, systems
\end{IEEEkeywords}

\section{Introduction}


\printbibliography

\end{document}